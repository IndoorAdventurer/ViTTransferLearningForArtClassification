\section{Methods}

% Experiments build upon Sabatelli et al.
Most of the experiments and analysis discussed below build upon the work done in \citeauthor{sabatelli2018deep} (\citeyear{sabatelli2018deep}).

\subsection{Dataset}
% Mention the challenges and show the table, of course. Also add max and min classes per category to the table maybe
% Don't forget to tell that images get scaled and cropped to 224 * 224 size and imagenet normalised
% Dataset described in more detail in Appendix A, including histograms and confusion matrices + balancing 'algorithm'

\begin{table*}[tb]
\centering
\small
\begin{tabular}{lllll}
\hline
\textbf{Task} & \textbf{\# Samples} & \textbf{\# Classes} & \textbf{Gini coefficient} & \textbf{Sample overlap} \\ \hline
Type classification & 9607 (77628) & 30 (801) & 0.466 (0.974) & 0.686 \\
Material classification & 7788 (96583) & 30 (136) & 0.563 (0.980) & 0.798 \\
Artist classification & 6530 (38296) & 30 (8592) & 0.236 (0.676) & 1 \\
Scaling experiment & 7926 (77628) & 15 (801) & 0.300 (0.974) & - \\ \hline
\end{tabular}
\caption{Overview of the used datasets. Values between brackets show the situation before balancing operations were performed. `Sample overlap' gives the average overlap between 2 of the 5 randomly generated sets per task ($i$ and $j$ where $i \neq j$).}
\label{methods:datasets}
\end{table*}

\subsection{Models}
% General information (4 per type, base/tiny etc)
% Paragraph/section about CNN models
%  - ResNet50 and VGG19 best fine-tuned and ots networks in Sabatelli
%  - ResNet50 I think also used in medical imaging paper; ResNet101 ots vit paper
%  * Convnext and EfficientNetV2 to have some more modern ones in the mix (to make comparison more fair)
% Paragraph/section about ViT models -- explain the idea behind each in 1 sentence maybe (also for CNNs)

\subsection{Hyperparameters}


\subsection{Hardware and software}
